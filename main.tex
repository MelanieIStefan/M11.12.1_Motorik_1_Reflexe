\documentclass{beamer}	
\mode<presentation>
 
\usepackage{pdfpages}
\usepackage{fancyvrb}
\usepackage{chemarr}

\usepackage{amsmath}		%% mathematics typesetting
\usepackage{amssymb}
 
\usepackage{epigraph}   %% nice setting of quotations

\usepackage{tabularx} %% allows to use row colours in tables

\usepackage{ulem}

\usepackage{booktabs}

\usepackage{siunitx} %% tpyeset SI units

\usepackage{CJKutf8} %% typeset Chinese characters

\usepackage{pdfpages}%% include pdfs

\usepackage{graphicx}
\usepackage{animate} %% show animated gifs

\DeclareMathAlphabet{\mathcalligra}{T1}{calligra}{m}{n}


% Color and Theme. Can be changed. However, this one's quite nice.
\usetheme{Madrid}
\definecolor{theme}{rgb}{0.84,0,0.21}
\usecolortheme[named=theme]{structure}

%%  Title information
% \title[M11.11.5 Lernen, Gedächtnis]{M11.11.5 Zentrales Nervensystem 3: \\ Lernen und Gedächtnis}
% \author[melanie.stefan@medicalschool-berlin.de]{}
% \institute[]{Prof. Melanie Stefan \\ melanie.stefan@medcialschool-berlin.de}
% \date{SoSe 2022}
 

% Table of contents to pop up at the beginning of each section
\AtBeginSection[]
{
  \begin{frame}<beamer>
    \frametitle{Outline}
    \tableofcontents[currentsection,currentsubsection]
  \end{frame}
}
 
\beamertemplatenavigationsymbolsempty

\begin{document}


{ \usebackgroundtemplate{\includegraphics[width=1\paperwidth]{MSB_Titelseite.pdf}} 
\begin{frame}

 \maketitle 

$\,$\\[6cm] 


\end{frame} 
}


%% Hook:

% { \usebackgroundtemplate{\includegraphics[width=1.2\paperwidth]{miklos-kornyei-pWzCAm4jgzA-unsplash.jpg}} 
% \begin{frame}

% \end{frame}
% }

 
%% %% TLIA



%% %% Learning Objectives
 
\begin{frame}

 \frametitle{Nach dieser Vorlesung sollten Sie folgendes können}



\begin{block}{Grundlagen:}
\begin{itemize}
\item
\end{itemize}

\end{block}



\end{frame}


 
\begin{frame}

 \frametitle{Nach dieser Vorlesung sollten Sie folgendes können}

\begin{block}{Klinik:}
\begin{itemize}
\item

\end{itemize}

\end{block}



\end{frame}









%% %% %% Main Body
 

\section{Grundlegende Definitionen}


%% FINAL: Comment from here
\begin{frame}
\frametitle{Grundlegene Definitionen}

\textcolor{theme}{Ordnen Sie zu!}

\begin{columns}[c]


\begin{column}{3cm}

\begin{tabular}{|l|}
\hline 
Motorik     \\
\hline 
Reflex \\
\hline 
Willkürbewegung \\
\hline 
Haltung \\
\hline 
Zielmotorik \\
     \hline 
\end{tabular}

\end{column}


\begin{column}{7cm}

\begin{tabular}{|p{7cm}|}
\hline
Muskelkontraktionen, die Aufrichten des Körpers gegen die Schwerkraft ermöglichen \\
\hline
von kortikalen und anderen hohen ZNS-Strukturen initierte Bewegungen \\
\hline
Aktivierung von Muskeln durch ZNS \\
\hline
automatische und unwillkürliche Antwort der Muskulatur auf Rezeptorerregung \\
\hline
Muskelkontraktionen, die zielgerichtet sind \\
\hline
\end{tabular}

\end{column}

\end{columns}



\end{frame}

%% FINAL: Comment to here



\begin{frame}
\frametitle{Grundlegene Definitionen}



\begin{tabular}{|l|p{7cm}|}
\hline 
Motorik     & Aktivierung von Muskeln durch ZNS \\
\hline 
Reflex & automatische und unwillkürliche Antwort der Muskulatur auf Rezeptorerregung \\
\hline 
Willkürbewegung & von kortikalen und anderen hohen ZNS-Strukturen initierte Bewegungen \\
\hline 
Haltung & Muskelkontraktionen, die Aufrichten des Körpers gegen die Schwerkraft ermöglichen \\
\hline 
Zielmotorik & Muskelkontraktionen, die zielgerichtet sind \\
     \hline 
\end{tabular}

\end{frame}




\section{Muskel}

%% Skelettmuskel: Grundlegendes: Einzig willkürlich kontrollierbar, aber nicht ausschliesslich  06:55

%% Skelettmuskel: Myofibrillen: Regulation, cna I find a good diagramm 06:55





\section{Vestibularsystem}
 
 \section{Cerebellum}

 \section{Rückenmark}
 
 \section{Reflexe}
 
 


%% %% %% %% Review

\begin{frame}

 \frametitle{Jetzt* sollten Sie folgendes können}



\end{frame}




%% %% %% %% Feedbackhinweisblock

\begin{frame}
\frametitle{Danke für Ihr Feedback!}

\begin{columns}[c]

\begin{column}{6cm}
\begin{center}
 \includegraphics[width=\textwidth]{smilie_balloons.jpg}
\end{center}

\end{column}

\begin{column}{4cm}


\begin{center}
% \includegraphics[width=\textwidth]{feedback_QR.png}
\end{center}
\end{column}


\end{columns}
\end{frame}


%% %% %% Bildnachweis
\begin{frame}
\frametitle{Bildnachweis}
\begin{tiny}

% Teile dieser Vorlesung wurden übernommen von einer Vorlesung von Prof. Emanuel Busch,  Health and Medical University Potsdam, dem wir an dieser Stelle herzlich danken. Wo nicht anders gekennzeichnet, stammen Abbildungen aus dieser Vorlesung.  


 
\begin{itemize}



%% all lectures
\item
Luftballons mit frohen und traurigen Smilies. Photo by \href{https://unsplash.com/@artbyhybrid?utm_source=unsplash&utm_medium=referral&utm_content=creditCopyText}{Hybrid} on \href{https://unsplash.com/s/photos/feedback?utm_source=unsplash&utm_medium=referral&utm_content=creditCopyText}{Unsplash}
%%%%%%%%%%%




\end{itemize}
\end{tiny}
\end{frame}






\end{document}

%%% Frequently used snippets

%% \begin{columns}[c]

%% \begin{column}{5cm}
%% \end{column}

%% \begin{column}{5cm}
%% \end{column}


%% \end{columns}




