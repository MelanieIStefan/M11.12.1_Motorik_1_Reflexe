\documentclass{beamer}	
\mode<presentation>
 
\usepackage{pdfpages}
\usepackage{fancyvrb}
\usepackage{chemarr}

\usepackage{amsmath}		%% mathematics typesetting
\usepackage{amssymb}
 
\usepackage{epigraph}   %% nice setting of quotations

\usepackage{tabularx} %% allows to use row colours in tables

\usepackage{ulem}

\usepackage{booktabs}

\usepackage{siunitx} %% tpyeset SI units

\usepackage{CJKutf8} %% typeset Chinese characters

\usepackage{pdfpages}%% include pdfs

\usepackage{graphicx}
\usepackage{animate} %% show animated gifs

\DeclareMathAlphabet{\mathcalligra}{T1}{calligra}{m}{n}


% Color and Theme. Can be changed. However, this one's quite nice.
\usetheme{Madrid}
\definecolor{theme}{rgb}{0.84,0,0.21}
\usecolortheme[named=theme]{structure}

%%  Title information
\title[M11.12.1 Reflexe]{M11.12.1 Motorik 1: \\ Reflexe}
\author[melanie.stefan@medicalschool-berlin.de]{}
\institute[]{Prof. Melanie Stefan \\ melanie.stefan@medcialschool-berlin.de}
\date{SoSe 2022}
 

% Table of contents to pop up at the beginning of each section
\AtBeginSection[]
{
  \begin{frame}<beamer>
    \frametitle{Outline}
    \tableofcontents[currentsection,currentsubsection]
  \end{frame}
}
 
\beamertemplatenavigationsymbolsempty

\begin{document}


{ \usebackgroundtemplate{\includegraphics[width=1.2\paperwidth]{MSB_Titelseite.pdf}} 
\begin{frame}

 \maketitle 

$\,$\\[6cm] 


\end{frame} 
}


%% Hook:

{ \usebackgroundtemplate{\includegraphics[width=1.2\paperwidth]{Arztkoffer.jpg}} 
\begin{frame}
\pause
\begin{flushright}
\textcolor{yellow}{Was macht der Hammer?}
\end{flushright}

$\,$\\[9cm]



\end{frame}
}

 
%% %% TLIA
\begin{frame}{In dieser Vorlesung geht es um \dots}

\begin{columns}[c]

\begin{column}{5cm}
\begin{center}
    \includegraphics[width=\textwidth]{Reflexhammer.jpg}
\end{center}
\end{column}

\begin{column}{5cm}

Grundlagen der Motorik, und vor allem: Reflexe, und wie sie zustande kommen

\end{column}


\end{columns}

\end{frame}


%% %% Learning Objectives
 
\begin{frame}

 \frametitle{Nach dieser Vorlesung sollten Sie folgendes können}



\begin{block}{Grundlagen:}
\begin{itemize}
\item
\end{itemize}

\end{block}



\end{frame}


 
\begin{frame}

 \frametitle{Nach dieser Vorlesung sollten Sie folgendes können}

\begin{block}{Klinik:}
\begin{itemize}
\item

\end{itemize}

\end{block}



\end{frame}









%% %% %% Main Body
 

\section{Grundlegende Definitionen}


%% FINAL: Comment from here
\begin{frame}
\frametitle{Grundlegene Definitionen}

\textcolor{theme}{Ordnen Sie zu!}

\begin{columns}[c]


\begin{column}{3cm}

\begin{tabular}{|l|}
\hline 
Motorik     \\
\hline 
Reflex \\
\hline 
Willkürbewegung \\
\hline 
Haltung \\
\hline 
Zielmotorik \\
     \hline 
\end{tabular}

\end{column}


\begin{column}{7cm}

\begin{tabular}{|p{7cm}|}
\hline
Muskelkontraktionen, die Aufrichten des Körpers gegen die Schwerkraft ermöglichen \\
\hline
von kortikalen und anderen hohen ZNS-Strukturen initierte Bewegungen \\
\hline
Aktivierung von Muskeln durch ZNS \\
\hline
automatische und unwillkürliche Antwort der Muskulatur auf Rezeptorerregung \\
\hline
Muskelkontraktionen, die zielgerichtet sind \\
\hline
\end{tabular}

\end{column}

\end{columns}



\end{frame}

%% FINAL: Comment to here



\begin{frame}
\frametitle{Grundlegene Definitionen}



\begin{tabular}{|l|p{7cm}|}
\hline 
Motorik     & Aktivierung von Muskeln durch ZNS \\
\hline 
Reflex & automatische und unwillkürliche Antwort der Muskulatur auf Rezeptorerregung \\
\hline 
Willkürbewegung & von kortikalen und anderen hohen ZNS-Strukturen initierte Bewegungen \\
\hline 
Haltung & Muskelkontraktionen, die Aufrichten des Körpers gegen die Schwerkraft ermöglichen \\
\hline 
Zielmotorik & Muskelkontraktionen, die zielgerichtet sind \\
     \hline 
\end{tabular}

\end{frame}




\section{Muskel}

%% Skelettmuskel: Grundlegendes: Einzig willkürlich kontrollierbar, aber nicht ausschliesslich  06:55
\begin{frame}{Erinnerung: Muskulatur}
    
\begin{itemize}
    \item 
    Wir haben 3 Arten von Muskulatur (Herzmuskulatur, glatte Muskulatur, Skelettmuskulatur)
    \item
    Nur Sekelettmuskulatur kann willkürlich bewegt werden
    \item
    Aber: Skelettmuskulatur kann auch unbewusst bewegt werden (Reflexe!)
    
\end{itemize}
\end{frame}



%% Skelettmuskel: Myofibrillen: Regulation, can I find a good diagramm 06:55
\begin{frame}{Aufbau von Skelettmuskeln}

\begin{center}
    \includegraphics[width=0.8\textwidth]{Skeletal_muscle.jpg}
\end{center}

    
\end{frame}




%% Kontraktion Aktivierung: Bild von Maike Vorlesung
\begin{frame}{Aufbau von Myofibrillen}


\begin{itemize}
    \item 
    Myofilamente: Aktin, Myosin
    \item
    Regulierende Proteine: Troponin, Tropomyosin
    \item
    Ruhezustand: Troponin und Tropomyosin hemmen Bindung von Aktin and Myosin
    \item
    Aktivierung durch Calcium: 
    \begin{itemize}
        \item 
            Aktin: Calcium bindet an Troponin, durch die entstehende Konformationsänderung kann Myosin an Aktin binden
            \item
            Myosin: Calcium aktiviert Myosin Light Chain Kinase (MLCK), durch die Phosphorylierung der MLC wird die Kontraktion effizienter
    \end{itemize}
    \item
    ATP Hydrolyse liefert die Energie für Kontraktion
\end{itemize}


\end{frame}


% %% Skelettmuskel: Myofibrillen: Regulation, can I find a good diagramm 06:55
%% Use pictues at https://de.wikipedia.org/wiki/Myosin
\begin{frame}{Aktin und Myosin interagieren, um den Muskel zu kontrahieren}
    
    \begin{center}
        \includegraphics<1>[width=0.6\textwidth]{Cross-bridge-cycle-4.png}
        
        \includegraphics<2>[width=0.6\textwidth]{Cross-bridge-cycle-1.png}
                
        \includegraphics<3>[width=0.6\textwidth]{Cross-bridge-cycle-2.png}
                        
        \includegraphics<4>[width=0.6\textwidth]{Cross-bridge-cycle-3.png}
    
        \includegraphics<5>[width=0.6\textwidth]{Cross-bridge-cycle-4.png}

    \end{center}
    
\end{frame}

\begin{frame}{Aktin und Myosin interagieren, um den Muskel zu kontrahieren}

    \begin{center}
        \includegraphics[width=\textwidth]{aktin_myosin_zyklus.png}
\end{center}

\end{frame}



%% FINAL version: comment from here
\begin{frame}{Ca\textsuperscript{2+}-Anstieg bei Aktivierung des Skelettmuskels}



\begin{columns}[c]

\begin{column}{5cm}

Aktionspotential im \(\alpha\) Motorneuron (1) führt zur Freisetzung von Acetylcholin (3) an der motorischen Endplatte. Im Sarkolemma der Muskelzelle (2) sitzen nACh Rzeptoren. Diese sind \textcolor{theme}{ionotrop oder metabotrop?}

\end{column}

\begin{column}{5cm}

\begin{center}
\includegraphics[width=\textwidth]{NMJ.png}    
\end{center}


\end{column}


\end{columns}

\end{frame}

%% FINAL version: comment to here


% \begin{frame}{Ca\textsuperscript{2+}-Anstieg bei Aktivierung des Skelettmuskels}



% \begin{columns}[c]

% \begin{column}{5cm}

% Aktionspotential im \(\alpha\) Motorneuron (1) führt zur Freisetzung von Acetylcholin (3) an der motorischen Endplatte. Im Sarkolemma der Muskelzelle (2) sitzen nACh Rzeptoren. Diese sind \textcolor{theme}{ionotrop oder metabotrop?}

% \end{column}

% \begin{column}{5cm}

% \begin{center}
% \includegraphics[width=\textwidth]{NMJ.png}    
% \end{center}


% \end{column}


% \end{columns}




% \end{frame}



%% Ca release from ER
% https://commons.wikimedia.org/wiki/File:Coupling_of_the_muscle_action_potential_to_Ca%2B%2B_release_from_the_sarcoplasmic_reticulum_-_the_dihydropyridine_receptor_and_ryanodine_receptor.png




\section{Vestibularsystem}
 
 \section{Cerebellum}

 \section{Rückenmark}
 
 \section{Reflexe}
 
 


%% %% %% %% Review

\begin{frame}

 \frametitle{Jetzt* sollten Sie folgendes können}



\end{frame}




%% %% %% %% Feedbackhinweisblock

\begin{frame}
\frametitle{Danke für Ihr Feedback!}

\begin{columns}[c]

\begin{column}{6cm}
\begin{center}
 \includegraphics[width=\textwidth]{smilie_balloons.jpg}
\end{center}

\end{column}

\begin{column}{4cm}


\begin{center}
\includegraphics[width=\textwidth]{feedback_QR.png}
\end{center}
\end{column}


\end{columns}
\end{frame}


%% %% %% Bildnachweis
\begin{frame}
\frametitle{Bildnachweis}
\begin{tiny}

Teile dieser Vorlesung wurden übernommen von einer Vorlesung von Prof. Maike Glitsch, Medical School Hamburg, der wir an dieser Stelle herzlich danken. Wo nicht anders gekennzeichnet, stammen Abbildungen aus dieser Vorlesung.  


 
\begin{itemize}

\item
Aufbau der Skelettmuskulatur. Montage von Raul654. CC BY-SA 3.0, \url{https://commons.wikimedia.org/w/index.php?curid=73744}

\item

Interaktion von Aktin und Myosin im Muskel.         Von Almut Hampl - via E-Mail, CC BY-SA 4.0, \url{https://commons.wikimedia.org/w/index.php?curid=107054691}

%% all lectures
\item
Luftballons mit frohen und traurigen Smilies. Photo by \href{https://unsplash.com/@artbyhybrid?utm_source=unsplash&utm_medium=referral&utm_content=creditCopyText}{Hybrid} on \href{https://unsplash.com/s/photos/feedback?utm_source=unsplash&utm_medium=referral&utm_content=creditCopyText}{Unsplash}
%%%%%%%%%%%

\item
Motorische Endplatte. CC BY-SA 3.0, \url{https://commons.wikimedia.org/w/index.php?curid=305017}


\item
Reflexhammer. By Polarlys - Own work, CC BY-SA 4.0, \url{https://commons.wikimedia.org/w/index.php?curid=74272543}

\item
Spielzeug-Arztkoffer. Cristiano Betta on Flickr,  \url{https://www.flickr.com/photos/cristiano_betta/432834207}, CC-BY 2.0, 2007.


\end{itemize}
\end{tiny}
\end{frame}






\end{document}

%%% Frequently used snippets

%% \begin{columns}[c]

%% \begin{column}{5cm}
%% \end{column}

%% \begin{column}{5cm}
%% \end{column}


%% \end{columns}




